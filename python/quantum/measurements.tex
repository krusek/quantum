\documentclass{article}
  \usepackage{amssymb}
  \usepackage{amsfonts}
  \usepackage{amsmath}
  \usepackage{amsthm}
  \usepackage[all]{xy}
  \usepackage{ifthen}
  \usepackage{makeidx}
  \usepackage{physics}
  \usepackage{exercise}
  \usepackage{tikz}

  % TikZ libraries `calc` needed now to tweak bracket.
  \usetikzlibrary{backgrounds,fit,decorations.pathreplacing,calc}

  \newcommand{\makeind}{\makeindex } \newcommand{\ind}[1]{ } \newcommand{\printind} {}
  %\usepackage{hyperref} \hypersetup{backref,colorlinks=true} \renewcommand{\ind}[1]{\index{#1}} \renewcommand{\makeind}{\makeindex} \renewcommand{\printind}{\printindex }
  \makeind
  \author{Korben Rusek}
  \title{Quantum Computing}
  \date{12-20-2018}
  \pagestyle{myheadings}
  \markright{Korben Rusek - Quantum Computing}
  \oddsidemargin 0.1in
  \evensidemargin 0.0in
  \textwidth 6.0in
  \begin{document}
  \maketitle
  \newcommand{\gindex}[2]{|#1\!:\!#2|}
  \newcommand{\lcm}{\textrm{lcm}}
  \newcommand{\irr}{\textrm{irr}}
  \newcommand{\sylp}{$Syl_{p}$}
  \newcommand{\phnt}[1]{$\phantom{1}^{#1}$}
  \newcommand{\gen}[1]{\langle#1\rangle}
  \newcommand{\BN}{\mathbb{N}}
  \newcommand{\BZ}{\mathbb{Z}}
  \newcommand{\BQ}{\mathbb{Q}}
  \newcommand{\BR}{\mathbb{R}}
  \newcommand{\BC}{\mathbb{C}}
  \newcommand{\BF}{\mathbb{F}}
  \newcommand{\CF}{\mathcal{F}}
  \newcommand{\CQ}{\mathcal{Q}}
  \newcommand{\fa}{\mathfrak{a}}
  \newcommand{\fb}{\mathfrak{b}}
  \newcommand{\fp}{\mathfrak{p}}
  \newcommand{\fq}{\mathfrak{q}}
  \newcommand{\fm}{\mathfrak{m}}
  \newcommand{\FN}{\mathfrak{N}}
  \newcommand{\FR}{\mathfrak{R}}
  \newcommand{\set}[1]{\{#1\}}
  \newcommand{\trv}{\set{1}}
  \newcommand{\Aut}{\mathrm{Aut}}
  \newcommand{\End}{\mathrm{End}}
  \newcommand{\Ker}{\mathrm{Ker}}
  \newcommand{\chr}{\mathrm{char}}

  \theoremstyle{definition}
  \newtheorem{theorem}{Theorem}[section]
  \newtheorem{corollary}[theorem]{Corollary}
  \newtheorem{definition}[theorem]{Definition}
  \newtheorem{lemma}[theorem]{Lemma}
  \newtheorem{fact}[theorem]{Fact}
  \newtheorem{exercise}{Exercise}[section]
  \newtheorem{example}{Example}[section]

  \setcounter{section}{1}
  \setcounter{subsection}{1}

\subsection{Joint Measurements}

The Pauli operators and how their measurements act on a 
set of qubits are well known. But most literature explains
how to modify the qubits such that you can perform the
Pauli $Z$ measurement on just the first qubit to achieve
the same result. 

There is a table containing tensor products of Pauli matrices
and operators that convert that operation to the simple $Z$
measurement and operators that convert them back.

The only issue with this method is that
it is harder to deal with the way that I've written my
quantum simulator. This document will outline how my 
simulator performs joint measurements and prove that it 
achieves the same results. It is not complicated, but I 
felt the need to more thoroughly document it.


\begin{fact}
  Pauli operators have eigenvalues $\pm1$ that split the
  quantum space into two equal subspaces.
\end{fact}

\begin{example}
  Let 
  \[
    \ket\psi= \frac{1}{2}\ket{100} 
      +\frac{\sqrt{2}}{2}\ket{011}
      +\frac{1}{2}\ket{101}.
  \]
  The Pauli $Z$ gate has the following action:
  \begin{eqnarray*}
    \ket{0}&\rightarrow&\ket{0}\\
    \ket{1}&\rightarrow&-\ket{1}
  \end{eqnarray*}
  That is $\ket0$ has $1$ as its eigenvalue and $\ket1$ has
  $-1$ as its eigenvalue. Therefore if we want to measure
  $\psi$ with respect to $Z\oplus I\oplus I$ it gets split as
  \[
    \frac{1}{2}\ket{100} 
      +\frac{1}{2}\ket{101}
  \]
  and
  \[
    \frac{\sqrt{2}}{2}\ket{011}.
  \]
  Looking at the coefficients each has a probability of $1/2$.
  Of course after measurement we need to normalize 
  (divide by the norm) the result in order to get a valid
  qubit array.

  Similarly measuring with respect to $I\oplus I\oplus Z$ will
  split $\ket\psi$ as
  \[
    \frac{1}{2}\ket{100}
  \]
  and
  \[
    \frac{\sqrt{2}}{2}\ket{011}
  +\frac{1}{2}\ket{101},
  \]
  The former with probability $1/4$ and the latter with probability
  $3/4$.
\end{example}

\begin{lemma}
  Let $\ket{\psi}$ be a quantum state and $\varphi$ a Pauli operator. 
  Suppose $\ket{\psi}$ can be written $\ket\psi=\ket{\alpha_+}+\ket{\alpha_-}$,
  where $\alpha_+$ has eigenvalue $1$ and $\alpha_-$ has eigenvalue $-1$,
  then $\ket{\alpha_+}$ and $\ket{\alpha_-}$ are unique.
  \begin{proof}
    First assume that
    $\ket\psi$ can be written two different ways,
    $\ket\psi=\ket{\alpha_{+}}+\ket{\alpha_{-}}$ and 
    $\ket\psi=\ket{\beta_+}+\ket{\beta_{-}}$ where
    $\ket{\alpha_+}$ and $\ket{\beta_+}$ have eigenvalue $1$ and
    $\ket{\alpha_{-}}$ and $\ket{\beta_{-}}$ have eigenvalue $-1$.
    I claim that $\ket{\alpha_+}=\ket{\beta_+}$ and 
    $\ket{\alpha_-}=\ket{\beta_-}$.
    Since their sum is $\ket\psi$ then we necessarily have
    \begin{eqnarray*}
      \ket{\alpha_+}-\ket{\beta_+}&=&\ket{\beta_-}-\ket{\alpha_-}\\
      \varphi\left(\ket{\alpha_+}-\ket{\beta_+}\right)&=&\varphi\left(\ket{\beta_-}-\ket{\alpha_-}\right)\\
      \ket{\alpha_+}-\ket{\beta_+}&=&\ket{\alpha_-}-\ket{\beta_-}.
    \end{eqnarray*}
    That is, $\ket{\alpha_+}-\ket{\beta_+}=-(\ket{\alpha_+}-\ket{\beta_+})$ and
    so we must have $\ket{\alpha_+}-\ket{\beta_+}=0$ and $\ket{\alpha_+}=\ket{\beta_+}$.
    Similarly we have $\ket{\alpha_-}=\ket{\beta_-}$.
  \end{proof}
\end{lemma}

This means that if we find a way to write our quantum state as the sum of
a vector with eigenvalue 1 and a vector with eigenvalue $-1$ then we are 
done.

\begin{lemma}
  Let $\ket{\psi}$ be a state and $\varphi$ any Pauli operator. Then
  $(\ket\psi+\varphi\ket\psi)/2$ has eigenvalue $1$ or is $0$. Similarly,
  $(\ket\psi-\varphi\ket\psi)/2$ has eigenvalue $-1$ or is $0$.
  \begin{proof}
    The proof is pretty straightforward. Assume that $\ket\psi+\varphi\ket\psi\ne0$.
    Then
    \begin{eqnarray*}
      \varphi\left(\ket\psi+\varphi\ket\psi\right) 
      &=&\varphi(\ket\psi)+\varphi(\varphi(\ket\psi)) \\
      &=&\varphi(\ket\psi)+\ket\psi \\
      &=&\ket\psi+\varphi\ket\psi.
    \end{eqnarray*}
    Hence it has eigenvalue $1$.
    
    Similarly if $\ket\psi-\varphi\ket\psi\ne0$,
    \begin{eqnarray*}
      \varphi\left(\ket\psi-\varphi\ket\psi\right) 
      &=&\varphi(\ket\psi)-\varphi(\varphi(\ket\psi)) \\
      &=&\varphi(\ket\psi)-\ket\psi \\
      &=&-(\ket\psi-\varphi(\ket\psi).
    \end{eqnarray*}
    Hence it has eigenvalue $-1$.
  \end{proof}
\end{lemma}

This is how measurements are performed in my quantum emulator.

% \begin{fact}
%   A Pauli measurement splits a set of qubits between the
%   positive eigenvalue parts and the negative eigenvalue parts
%   then based on the probabilities will 
% \end{fact}

\end{document}
