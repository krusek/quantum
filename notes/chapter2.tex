\documentclass{article}
\usepackage{amssymb}
\usepackage{amsfonts}
\usepackage{amsmath}
\usepackage{amsthm}
\usepackage[all]{xy}
\usepackage{ifthen}
\usepackage{makeidx}
\usepackage{physics}
\usepackage{exercise}
\newcommand{\makeind}{\makeindex } \newcommand{\ind}[1]{ } \newcommand{\printind} {}
%\usepackage{hyperref} \hypersetup{backref,colorlinks=true} \renewcommand{\ind}[1]{\index{#1}} \renewcommand{\makeind}{\makeindex} \renewcommand{\printind}{\printindex }
\makeind
\author{Korben Rusek}
\title{Quantum Computing}
\date{9-5-2007}
\pagestyle{myheadings}
\markright{Korben Rusek - Chapter 2 Notes}
\oddsidemargin 0.1in
\evensidemargin 0.0in
\textwidth 6.0in
\begin{document}
\maketitle
\newcommand{\gindex}[2]{|#1\!:\!#2|}
\newcommand{\lcm}{\textrm{lcm}}
\newcommand{\irr}{\textrm{irr}}
\newcommand{\sylp}{$Syl_{p}$}
\newcommand{\phnt}[1]{$\phantom{1}^{#1}$}
\newcommand{\gen}[1]{\langle#1\rangle}
\newcommand{\BN}{\mathbb{N}}
\newcommand{\BZ}{\mathbb{Z}}
\newcommand{\BQ}{\mathbb{Q}}
\newcommand{\BR}{\mathbb{R}}
\newcommand{\BC}{\mathbb{C}}
\newcommand{\BF}{\mathbb{F}}
\newcommand{\CF}{\mathcal{F}}
\newcommand{\CQ}{\mathcal{Q}}
\newcommand{\fa}{\mathfrak{a}}
\newcommand{\fb}{\mathfrak{b}}
\newcommand{\fp}{\mathfrak{p}}
\newcommand{\fq}{\mathfrak{q}}
\newcommand{\fm}{\mathfrak{m}}
\newcommand{\FN}{\mathfrak{N}}
\newcommand{\FR}{\mathfrak{R}}
\newcommand{\set}[1]{\{#1\}}
\newcommand{\trv}{\set{1}}
\newcommand{\Aut}{\mathrm{Aut}}
\newcommand{\End}{\mathrm{End}}
\newcommand{\Ker}{\mathrm{Ker}}
\newcommand{\chr}{\mathrm{char}}
\theoremstyle{definition}
\newtheorem{theorem}{Theorem}[section]
\newtheorem{definition}[theorem]{Definition}


\section{The Pauli matrices}


\[
  \sigma_0=I
\]

\[
  \sigma_1=\sigma_x=X=
  \begin{bmatrix}
    0&1 \\
    1&0
  \end{bmatrix}
\]

\[
  \sigma_2=\sigma_y=Y=
  \begin{bmatrix}
    0&-i \\
    i&0
  \end{bmatrix}
\]

\[
  \sigma_3=\sigma_z=Z=
  \begin{bmatrix}
    1&0 \\
    0&-1
  \end{bmatrix}
\]

\section{Adjoints and Hermitian operators}

\begin{definition}[Hermitian]
  An operator $A$ is Hermitian if $A=A^\dagger$.
\end{definition}

\begin{theorem}
  Two eigenvectors of a Hermitian operator with different eigenvalues are orthogonal.
\end{theorem}

\begin{definition}
  A matrix is normal if $AA^\dagger=A^\dagger A$.
\end{definition}

\begin{theorem}
  A normal matrix is Hermitian iff it has real eigenvalues.
\end{theorem}

\begin{definition}[Unitary]
  A matrix $U$ is unitary if $U^\dagger U=I$.
\end{definition}

\begin{definition}[Positive and Positive definite]
  A positive operator $A$ is defined to be an operator
  such that for any vector $\ket{v}$, $(\ket{v},A\ket{v})$
  is a real, non-negative number. If $(\ket{v}, A\ket{v})>0$
  then $A$ is positive definite.
\end{definition}

\section{The polar and singular value decompositions}

\begin{theorem}[Polar decompositions]
  Let $A$ be a linear operator on a vector space $V$. Then
  there exists unitary $U$ and positive operators $J$ and $K$
  such that
  \[A=UJ=KU,\]
  where the unique positive operators $J$ and $K$ satisfying
  these equations are $J=\sqrt{A^\dagger A}$ and $K=\sqrt{AA^\dagger}$.
  Moreover, if $A$ is invertible then $U$ is unique.
\end{theorem}

\begin{theorem}[Singular value decomposition]
  Let $A$ be a square matrix. Then there exist unitary matrices
  $U$ and $V$, and a diagonal matrix $D$ with non-negative entries
  such that
  \[A=UDV.\]
  The diagonal elements of $D$ are known as the
  \it{singular values} of $A$.
\end{theorem}

\end{document}
