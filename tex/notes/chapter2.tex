\documentclass{article}
\usepackage{amssymb}
\usepackage{amsfonts}
\usepackage{amsmath}
\usepackage{amsthm}
\usepackage[all]{xy}
\usepackage{ifthen}
\usepackage{makeidx}
\usepackage{physics}
\usepackage{exercise}
\newcommand{\makeind}{\makeindex } \newcommand{\ind}[1]{ } \newcommand{\printind} {}
%\usepackage{hyperref} \hypersetup{backref,colorlinks=true} \renewcommand{\ind}[1]{\index{#1}} \renewcommand{\makeind}{\makeindex} \renewcommand{\printind}{\printindex }
\makeind
\author{Korben Rusek}
\title{Quantum Computing}
\date{6-1-2018}
\pagestyle{myheadings}
\markright{Korben Rusek - Chapter 2 Notes}
\oddsidemargin 0.1in
\evensidemargin 0.0in
\textwidth 6.0in
\begin{document}
\maketitle
\newcommand{\gindex}[2]{|#1\!:\!#2|}
\newcommand{\lcm}{\textrm{lcm}}
\newcommand{\irr}{\textrm{irr}}
\newcommand{\sylp}{$Syl_{p}$}
\newcommand{\phnt}[1]{$\phantom{1}^{#1}$}
\newcommand{\gen}[1]{\langle#1\rangle}
\newcommand{\BN}{\mathbb{N}}
\newcommand{\BZ}{\mathbb{Z}}
\newcommand{\BQ}{\mathbb{Q}}
\newcommand{\BR}{\mathbb{R}}
\newcommand{\BC}{\mathbb{C}}
\newcommand{\BF}{\mathbb{F}}
\newcommand{\CF}{\mathcal{F}}
\newcommand{\CQ}{\mathcal{Q}}
\newcommand{\fa}{\mathfrak{a}}
\newcommand{\fb}{\mathfrak{b}}
\newcommand{\fp}{\mathfrak{p}}
\newcommand{\fq}{\mathfrak{q}}
\newcommand{\fm}{\mathfrak{m}}
\newcommand{\FN}{\mathfrak{N}}
\newcommand{\FR}{\mathfrak{R}}
\newcommand{\set}[1]{\{#1\}}
\newcommand{\trv}{\set{1}}
\newcommand{\Aut}{\mathrm{Aut}}
\newcommand{\End}{\mathrm{End}}
\newcommand{\Ker}{\mathrm{Ker}}
\newcommand{\chr}{\mathrm{char}}
\theoremstyle{definition}
\newtheorem{theorem}{Theorem}[section]
\newtheorem{definition}[theorem]{Definition}
\newtheorem{postulate}{Postulate}[section]

\section{Linear Algebra}

\setcounter{subsection}{2}
\subsection{The Pauli matrices}


\[
  \sigma_0=I
\]

\[
  \sigma_1=\sigma_x=X=
  \begin{bmatrix}
    0&1 \\
    1&0
  \end{bmatrix}
\]

\[
  \sigma_2=\sigma_y=Y=
  \begin{bmatrix}
    0&-i \\
    i&0
  \end{bmatrix}
\]

\[
  \sigma_3=\sigma_z=Z=
  \begin{bmatrix}
    1&0 \\
    0&-1
  \end{bmatrix}
\]

\setcounter{subsection}{5}
\subsection{Adjoints and Hermitian operators}

\begin{definition}[Hermitian]
  An operator $A$ is Hermitian if $A=A^\dagger$.
\end{definition}

\begin{theorem}
  Two eigenvectors of a Hermitian operator with different eigenvalues are orthogonal.
\end{theorem}

\begin{definition}
  A matrix is normal if $AA^\dagger=A^\dagger A$.
\end{definition}

\begin{theorem}
  A normal matrix is Hermitian iff it has real eigenvalues.
\end{theorem}

\begin{definition}[Unitary]
  A matrix $U$ is unitary if $U^\dagger U=I$.
\end{definition}

\begin{definition}[Positive and Positive definite]
  A positive operator $A$ is defined to be an operator
  such that for any vector $\ket{v}$, $(\ket{v},A\ket{v})$
  is a real, non-negative number. If $(\ket{v}, A\ket{v})>0$
  then $A$ is positive definite.
\end{definition}

\subsection{Tensor products}
\subsection{Operator functions}
\subsection{The commutator and anti-commutator}
\begin{definition}[Commutator]
  The \textit{commutator} between two operators $A$ and $B$ is defined to be
  \[[A,B]=AB-BA.\]
  If $[A,B]=0$ then we say that $A$ commutes with $B$.
\end{definition}
\begin{definition}[Anti-commutator]
  The \textit{anti-commutator} between to operators $A$ and $B$ is defined to be
  \[\{A,B\}=AB+BA.\]
  If $\{A,B\}=0$ then we say that $A$ anti-commutes with $B$.
\end{definition}

\begin{definition}[Simultaneously Diagonalizable]
  Hermitian operators $A$ and $B$ are said to be \textit{simultaneously diagonalizable} if there exist some orthonormal set of vectors $\bra{i}$ such that $A=\sum a_i\ket{i}\bra{i}$ and $B=\sum b_i\ket{i}\bra{i}$.
\end{definition}

\begin{theorem}[Simultaneous diagonalization theorem]
  Suppose $A$ and $B$ are Hermitian operators. Then $[A,B]=0$ iff $A$ and $B$ are simultaneously diagonalizable.
\end{theorem}

Here are some facts

\[
  [X,Y]=2iZ; [Y,Z]=2iX; [Z,X]=2iY
\]
\[
  [A,B]^\dagger = [B^\dagger, A^\dagger]
\]

\begin{theorem}
  Suppose $A$ and $B$ are Hermitian. Then $i[A,B]$ is also Hermitian.
\end{theorem}
\subsection{The polar and singular value decompositions}

\begin{theorem}[Polar decompositions]
  Let $A$ be a linear operator on a vector space $V$. Then
  there exists unitary $U$ and positive operators $J$ and $K$
  such that
  \[A=UJ=KU,\]
  where the unique positive operators $J$ and $K$ satisfying
  these equations are $J=\sqrt{A^\dagger A}$ and $K=\sqrt{AA^\dagger}$.
  Moreover, if $A$ is invertible then $U$ is unique.
\end{theorem}

\begin{theorem}[Singular value decomposition]
  Let $A$ be a square matrix. Then there exist unitary matrices
  $U$ and $V$, and a diagonal matrix $D$ with non-negative entries
  such that
  \[A=UDV.\]
  The diagonal elements of $D$ are known as the
  \textit{singular values} of $A$.
\end{theorem}

\section{The postulates of quantum mechanics}
\subsection{State Space}


\begin{postulate}
  Associated to any isolated physical system is a complex vector space with inner product (that is, a Hilbert space) known as the \textit{state space} of the system. The system is completely described by its \textit{state vector}, which is a unit vector in the system's state space.
\end{postulate}

\subsection{Evolution}

\begin{postulate}
  The evolution of a \textit{closed} quantum system is described by a \textit{unitary transformation}. That is, the state $\ket{\phi}$ of the system at time $t_1$ is related to the state $\ket{\phi'}$ of the system at time $t_2$ by a unitary operator $U$ which depends only on the times $t_1$ and $t_2$,
  \[\ket{\phi'}=U\ket{\phi}.\]
\end{postulate}

The $X$ Pauli matrix is often referred to as the not gate or the \textit{bit flip} gate. It will send $\ket0$ to $\ket1$ and $\ket1$ to $\ket0$.

The $Z$ Pauli matrix is called the \textit{phase flip} gate. It leaves $\ket0$ invariant but sends $\ket1$ to $-\ket1$.

\begin{definition}[Hadamard Gate]
  An interesting unitary operator is the \textit{Hadamard gate}. This had the matrix representation
  \[H=\frac{1}{\sqrt2}
  \begin{bmatrix}
    1&1 \\
    1&-1
  \end{bmatrix}\]
\end{definition}

\setcounter{postulate}{1}

\begin{postulate}[Revised]
  The time evolution of the state of a closed quantum system is described by the \textit{Schr\"odinger equation},
  \[i\hbar\frac{d\ket\phi}{dt}=H\ket\phi.\]
  In this equation, $\hbar$ is a physical constant known as \textit{Planck's constant} whose value must be experimentally determined. The exact value is not important to us. In practice, it is common to absorb the factor $\hbar$ into $H$, effectively setting $\hbar=1$. $H$ is a fixed Hermitian operator known as the \textit{Hamiltonian} of the closed system.
\end{postulate}

\subsection{Quantum measurement}

We introduce Postulate 3 in order to describe the effects that measurement have on a system.

\begin{postulate}
  Quantum measurements are described by a collection $\{M_m\}$ of \textit{measurement operators}. These are operators acing on the state space of the system being measured. The index $m$ refers to the measurement outcomes that may occur in the experiment. If the state of the quantum system is $\ket\phi$ immediately before the measurement then the probability that result $m$ occurs is given by
  \[p(m)=\bra{\phi}M_m^\dagger M_m\ket{\phi},\]
  and the state of the system after the measurement is
  \[\frac{M_m\ket{\phi}}{\sqrt{\bra\phi M_m^\dagger M_m\ket\phi}}.\]
  The measurement operators satisfy the \textit{completeness equation},
  \[\sum_m M_m^\dagger M_m = I.\]
\end{postulate}

The completeness equation expresses the fact that probabilities sum to one:
\[I=\sum p(m)=\sum \bra\phi M^\dagger_m M_m\ket\phi.\]

\end{document}
