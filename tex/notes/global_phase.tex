\documentclass{article}
\usepackage{amssymb}
\usepackage{amsfonts}
\usepackage{amsmath}
\usepackage{amsthm}
\usepackage[all]{xy}
\usepackage{ifthen}
\usepackage{makeidx}
\usepackage{physics}
\usepackage{exercise}
\newcommand{\makeind}{\makeindex } \newcommand{\ind}[1]{ } \newcommand{\printind} {}
%\usepackage{hyperref} \hypersetup{backref,colorlinks=true} \renewcommand{\ind}[1]{\index{#1}} \renewcommand{\makeind}{\makeindex} \renewcommand{\printind}{\printindex }
\makeind
\author{Korben Rusek}
\title{Quantum Computing}
\date{7-16-2019}
\pagestyle{myheadings}
\markright{Korben Rusek - Global Phase}
\oddsidemargin 0.1in
\evensidemargin 0.0in
\textwidth 6.0in
\begin{document}
\maketitle
\newcommand{\gindex}[2]{|#1\!:\!#2|}
\newcommand{\lcm}{\textrm{lcm}}
\newcommand{\irr}{\textrm{irr}}
\newcommand{\sylp}{$Syl_{p}$}
\newcommand{\phnt}[1]{$\phantom{1}^{#1}$}
\newcommand{\gen}[1]{\langle#1\rangle}
\newcommand{\BN}{\mathbb{N}}
\newcommand{\BZ}{\mathbb{Z}}
\newcommand{\BQ}{\mathbb{Q}}
\newcommand{\BR}{\mathbb{R}}
\newcommand{\BC}{\mathbb{C}}
\newcommand{\BF}{\mathbb{F}}
\newcommand{\CF}{\mathcal{F}}
\newcommand{\CQ}{\mathcal{Q}}
\newcommand{\fa}{\mathfrak{a}}
\newcommand{\fb}{\mathfrak{b}}
\newcommand{\fp}{\mathfrak{p}}
\newcommand{\fq}{\mathfrak{q}}
\newcommand{\fm}{\mathfrak{m}}
\newcommand{\FN}{\mathfrak{N}}
\newcommand{\FR}{\mathfrak{R}}
\newcommand{\set}[1]{\{#1\}}
\newcommand{\trv}{\set{1}}
\newcommand{\Aut}{\mathrm{Aut}}
\newcommand{\End}{\mathrm{End}}
\newcommand{\Ker}{\mathrm{Ker}}
\newcommand{\chr}{\mathrm{char}}
\theoremstyle{definition}
\newtheorem{theorem}{Theorem}[section]
\newtheorem{lemma}[theorem]{Lemma}
\newtheorem{corollary}[theorem]{Corollary}
\newtheorem{definition}[theorem]{Definition}
\newtheorem{postulate}{Postulate}[section]

\section{Global Phase}

\setcounter{subsection}{0}
\subsection{Global Phase}

Let's begin with a definition of global phase. It is a rather simple
concept. We can basically just say that global phase is multiplying 
a of qubit by a constant of length 1. It is important to distinguish 
this from multiplication in a controlled fashion. That is, multiplying 
qubit $q_0$ by $\alpha$ whenever $q_1$ is in the $\ket1$ state.

In the end we will show that a quantum computer cannot distinguish 
two qubits that differ by a global phase. That is to say, suppose you 
have two buckets of qubits. The qubits in the first bucket are all
the same and the qubits in the second bucket are all the same. Now 
someone tells you that all the qubits in the second bucket are either
the same as the first or differ by a global phase. There would be 
no way of knowing which of the two was the case.
The implication being that we can safely disregard global phase.

\subsection{Measurement}
Let's begin by a short discussion of measurement. We begin with
two qubits $\ket{a}$ and $\ket{b}$ that form a basis. This means 
that any qubit $\ket{q}$ can be written as a linear combination
$\ket{q}=a_0\ket{a}+b_0\ket{b}$, with $|a_0|^2+|b_0|^2=1$. Next
we have a measurement $M$ that for which $\ket{a}$ has eigenvalue 
$\alpha$ and $\ket{b}$ has eigenvalue $\beta$. This means that
upon measurement of $\ket{q}$ using $M$, we are given either 
$(\alpha,\ket{a})$ or $(\beta,\ket{b})$ with probabilities
$|a_0|^2$ and $|b_0|^2$, respectively. Note that the coefficients 
on $\ket{a}$ and $\ket{b}$ disappear entirely.

It is worth realizing that the concept of measurement is the 
\textit{only} way to dig into the inside of a qubit. Furthermore, 
it is the only way to distinguish between two qubits. This is 
a fundamental property of quantum physics.

We can make this more precise with a definition.

\begin{definition}[Measurably Distinct]
  Let $\ket{q_0}$ and $\ket{q_1}$ be two qubits. Let $\ket{a}$
  and $\ket{b}$ be a basis with eigenvalues $\alpha$ and $\beta$, 
  respectively. Let $M$ be a measurement that emits $\ket{a}$ 
  and $\ket{b}$. Finally suppose
  \begin{eqnarray*}
    \ket{q_0}&=&a_0\ket{a}+b_0\ket{b}\\
    \ket{q_1}&=&a_1\ket{a}+b_1\ket{b}.
  \end{eqnarray*}
  Then $\ket{q_0}$ and $\ket{q_1}$ are \textit{measureably distinct}
  if $|a_0|^2\ne|a_1|^2$ or $|b_0|^2\ne|b_1|^2$.
\end{definition}

If we have understood everything thus far, then our task to show
that global phase is inconsequential is to show that a global phase
change will not affect any measurement. That is, if the only way
for a quantum computer to differentiate two qubits is through 
measurement then we must show that a global phase has no bearing
on measurements.

We will start by showing that global phase has no bearing on 
$Z$ measurement.

\begin{lemma}
  Global phase has no bearing on $Z$ measurement.\label{lem:phasez}
  \begin{proof}
    This is very easy.
    Let $q=\alpha\ket{0}+\beta\ket{1}$. Now if you measure $q$
    then you get $\ket{0}$ with probability $|\alpha|^2$ and
    $\ket{1}$ with probability $|\beta|^2$. We will write that as
    \[
      M(q) = 
      \left\{
      \begin{matrix}
        \ket{0}&\quad&p=|\alpha|^2 \\
        \ket{1}&\quad&p=|\beta|^2
      \end{matrix}
      \right.
    \]

    Let $\gamma\in\BC$ such that $|\gamma|^2=1$. That is,
    $\gamma q$ differs from $q$ by a global phase. This means
    that $\gamma g = \gamma\alpha\ket0 + \gamma\beta\ket1$. 
    In this case measurement will give
    \[
      M(\gamma q) = 
      \left\{
      \begin{matrix}
        \ket{0}&\quad&p=|\gamma\alpha|^2=|\alpha|^2 \\
        \ket{1}&\quad&p=|\gamma\beta|^2=|\beta|^2
      \end{matrix}
      \right.
    \]
    That is the two qubits are not measurably distinct when
    we use a $Z$ measurement.
  \end{proof}
\end{lemma}

\begin{corollary}
  Global phase has no bearing on general measurements.\label{cor:phasegen}
  \begin{proof}
    Based on the proof of the above lemma it is easy to see how this follows.
  \end{proof}
\end{corollary}

\begin{corollary}
  Global phase has no bearing on measurements even when other 
  operations are involved.
  \begin{proof}
    Let $q$ be a qubit. Let $\gamma\in\BC$ such that $|\gamma|^2=1$.
    We will show that a chain of operators applied to $q$ is not
    measurably distinct from the same chain of operators applied to
    $\gamma q$. Let us start by simplifying the question to a single 
    operator applied to $q$ and $\gamma q$. We can make this assumption
    because the product of operators is yet another operator. Let 
    $N$ be any operator. Let $\psi = Nq$ be the qubit that results
    from applying $N$ to $q$. Then we have
    \begin{eqnarray*}
      N\gamma q &=& \gamma N q \\
      &=&\gamma\psi.
    \end{eqnarray*}
    Now by corollary \ref{cor:phasegen} we know that $\psi$ and 
    $\gamma\psi$ are not measurably distinct. Hence we have the 
    desired result.
  \end{proof}
\end{corollary}

\end{document}
