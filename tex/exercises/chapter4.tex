\documentclass{article}
  \usepackage{amssymb}
  \usepackage{amsfonts}
  \usepackage{amsmath}
  \usepackage{amsthm}
  \usepackage[all]{xy}
  \usepackage{ifthen}
  \usepackage{makeidx}
  \usepackage{physics}
  \usepackage{exercise}
  \usepackage{tikz}

  % TikZ libraries `calc` needed now to tweak bracket.
  \usetikzlibrary{backgrounds,fit,decorations.pathreplacing,calc}

  \newcommand{\makeind}{\makeindex } \newcommand{\ind}[1]{ } \newcommand{\printind} {}
  %\usepackage{hyperref} \hypersetup{backref,colorlinks=true} \renewcommand{\ind}[1]{\index{#1}} \renewcommand{\makeind}{\makeindex} \renewcommand{\printind}{\printindex }
  \makeind
  \author{Korben Rusek}
  \title{Quantum Computing}
  \date{6-1-2018}
  \pagestyle{myheadings}
  \markright{Korben Rusek - Quantum Computing}
  \oddsidemargin 0.1in
  \evensidemargin 0.0in
  \textwidth 6.0in
  \begin{document}
  \maketitle
  \newcommand{\gindex}[2]{|#1\!:\!#2|}
  \newcommand{\lcm}{\textrm{lcm}}
  \newcommand{\irr}{\textrm{irr}}
  \newcommand{\sylp}{$Syl_{p}$}
  \newcommand{\phnt}[1]{$\phantom{1}^{#1}$}
  \newcommand{\gen}[1]{\langle#1\rangle}
  \newcommand{\BN}{\mathbb{N}}
  \newcommand{\BZ}{\mathbb{Z}}
  \newcommand{\BQ}{\mathbb{Q}}
  \newcommand{\BR}{\mathbb{R}}
  \newcommand{\BC}{\mathbb{C}}
  \newcommand{\BF}{\mathbb{F}}
  \newcommand{\CF}{\mathcal{F}}
  \newcommand{\CQ}{\mathcal{Q}}
  \newcommand{\fa}{\mathfrak{a}}
  \newcommand{\fb}{\mathfrak{b}}
  \newcommand{\fp}{\mathfrak{p}}
  \newcommand{\fq}{\mathfrak{q}}
  \newcommand{\fm}{\mathfrak{m}}
  \newcommand{\FN}{\mathfrak{N}}
  \newcommand{\FR}{\mathfrak{R}}
  \newcommand{\set}[1]{\{#1\}}
  \newcommand{\trv}{\set{1}}
  \newcommand{\Aut}{\mathrm{Aut}}
  \newcommand{\End}{\mathrm{End}}
  \newcommand{\Ker}{\mathrm{Ker}}
  \newcommand{\chr}{\mathrm{char}}

  \theoremstyle{definition}
  \newtheorem{theorem}{Theorem}[section]
  \newtheorem{corollary}[theorem]{Corollary}
  \newtheorem{definition}[theorem]{Definition}
  \newtheorem{lemma}[theorem]{Lemma}
  \newtheorem{exercise}{Exercise}[section]

  \setcounter{section}{4}
  \setcounter{subsection}{1}

\subsection{Single qubit operators}

%4.1
\begin{exercise}
  In Exercise 2.11, you computed the eigenvectors of the Pauli matrices. Find the points on the Bloch sphere which correspond to the normalized eigenvectors of the different Pauli matrices.\label{ex:pauli-coordinates}
  \begin{proof}
    The eigenvectors of $X$ are $\frac{1}{\sqrt{2}}[1,1]$ and $\frac{1}{\sqrt{2}}[1,-1]$. These correspond to $[1,0,0]$ and $[-1,0,0]$, respectively. The eigenvectors of $Y$ are $\frac{1}{\sqrt2}[1,i]$ and $\frac{1}{\sqrt2}[1,-i]$. These correspond to $[0,1,0]$ and $[0,-1,0]$, respectively. Finally the eigenvectors, $[1,0]$ and $[0,1]$ of $Z$ correspond to $[0,0,1]$ and $[0,0,-1]$ respectively. That is, the eigenvectors of $X$, $Y$, and $Z$, unsurprisingly, correspond to the
    antipodal points along the $x$, $y$, and $z$-axes respectively.
  \end{proof}
\end{exercise}

\begin{lemma}
  Let $\ket{x}$ and $\ket{y}$ be orthogonal then the Bloch representations are antipodal. \label{lem:bloch}
  \begin{proof}

  \end{proof}
\end{lemma}

%4.2
\begin{exercise}
  Let $x$ be a real number and $A$ a matrix such that $A^2=I$. Show that
  \[\exp(Aix)=\cos(x)I+i\sin(x)A.\]
  User this result to verify Equations $(4.4)$ through $(4.6)$.
  \begin{proof}
    If $A^2=I$ then $A=\lambda_x\ket{x}\bra{x}+\lambda_y\ket{y}\bra{x}$ where $\lambda_x,\lambda_y=\pm1$. Therefore we have
    \begin{eqnarray*}
      \exp(iAx)&=&e^{ix\lambda_x}\ket{x}\bra{x}+e^{ix\lambda_y}\ket{y}\bra{y} \\
      &=& \left(\cos(x\lambda_x)+i\sin(x\lambda_x)\right)\ket{x}\bra{x}+\left(\cos(x\lambda_y)+i\sin(x\lambda_y)\right)\ket{y}\bra{y} \\
      &=& \left(\cos(x)+i\sin(x\lambda_x)\right)\ket{x}\bra{x}+\left(\cos(x)+i\sin(x\lambda_y)\right)\ket{y}\bra{y} \\
      &=& \left(\cos(x)+\lambda_xi\sin(x)\right)\ket{x}\bra{x}+\left(\cos(x)+\lambda_yi\sin(x)\right)\ket{y}\bra{y} \\
      &=& \cos(x)(\ket{x}\bra{x}+\ket{y}\bra{y})+\lambda_xi\sin(x)\ket{x}\bra{x}+\lambda_yi\sin(x)\ket{y}\bra{y} \\
      &=&\cos(x)I+i\sin(x)A.
    \end{eqnarray*}
    Where the $\lambda_x$ and $\lambda_y$ can be pulled out of the $\sin$ because they are $\pm1$.
  \end{proof}
\end{exercise}

%4.3
\begin{exercise}
  Show that, up to a global phase, the $\pi/8$ gate satisfies $T=R_x(\pi/4)$.
  \begin{proof}

  \end{proof}
\end{exercise}

%4.4
\begin{exercise}
  Express the Hadamard gate, $H$, as a product of $R_x$ and $R_z$ rotations and $e^{i\phi}$ for some $\phi$.
  \begin{proof}[Solution]
    First observe that $H\ket{0}=\ket{+}$ and $H\ket{+}=\ket{0}$. From exercise \ref{ex:pauli-coordinates} we know that $\ket{0}$ corresponds to the point $[0,0,1]$ and $\ket{+}$ corresponds to the point $[1,0,0]$. Now if we rotate $[0,0,1]$ along the $x$-axis by $\pi/2$ then we get the point $[0,1,0]$. Next if we rotate along the $z$-axis we get the point $[1,0,0]$. That is, $R_z(\pi/2)R_x(\pi/2)\ket{0}\simeq \ket{+}$. This gets us part of the way there.

    On the other hand, if we rotate the point $[1,0,0]$ ($\ket{+}$) along the $x$-axis by $\pi/2$ we get the point $[1,0,0]$ back. If we follow that by a rotation along the $z$-axis then we get the point $[0,-1,0]$. To get $[0,0,1]$ from this piont we need to follow this up with a rotation along the $x$-axis by $\pi/2$. That is, $R_x(\pi/2)R_z(\pi/2)R_x(\pi/2)\ket{+}\simeq \ket{0}$.

    Luckily it is also the case that $R_x(\pi/2)R_z(\pi/2)R_x(\pi/2)\ket{0}\simeq \ket{+}$. This means that
    \[R_x(\pi/2)R_z(\pi/2)R_x(\pi/2)= e^{i\alpha}H\]
    for some $\alpha\in\BR$. Multiplying out $R_x(\pi/2)R_z(\pi/2)R_x(\pi/2)$ we have
    \[
      \frac{1}{\sqrt2}\begin{bmatrix}
        -i&-i \\
        -i&i
      \end{bmatrix}.
    \]
    Therefore we just need to multiply by $e^{i\pi/2}$ and hence
    \[
      H=R_x(\pi/2)R_z(\pi/2)R_x(\pi/2)e^{i\pi/2}.
    \]

  \end{proof}
\end{exercise}

If $n=(n_x, n_y, n_z)\in\BR^3$ is a real unit vector in three dimensions then we genearlize the previous definitions by defining a rotation by $\theta$ about the $n$ axis by the equation
\begin{equation}
  R_n(\theta) = \exp(i\theta n\cdot\sigma/2)
  =\cos\left(\frac{\theta}{2}\right)I
  -i\sin\left(\frac{\theta}{2}\right)(n_xX+n_yY+n_zZ),\label{eq:rn}
\end{equation}
where $\sigma$ denotes the three component vector $(X,Y,Z)$ of Pauli matrices.

%4.5
\begin{exercise}
  Prove that $(n\cdot\sigma)^2=I$, and use this to verify equation \ref{eq:rn}.
  \begin{proof}
    Write $n=(x,y,z)$. Then we have
    \begin{eqnarray*}
      (n\cdot\sigma)^2&=&(xX+yY+zZ)^2 \\
      &=&x^2X^2+xyXY+xzXZ+y^2Y^2+xyYX+yzYZ+z^2Z^2+xzZX+yzZY \\
      &=&x^2I+xyXY+xzXZ+y^2I+xyYX+yzYZ+z^2I+xzZX+yzZY \\
      &=&(x^2+y^2+z^2)I+xyXY+xzXZ+xyYX+yzYZ+xzZX+yzZY \\
      &=&(x^2+y^2+z^2)I+xyXY-xyXY+xzXZ-xzXZ+yzYZ-yzYZ \\
      &=&(x^2+y^2+z^2)I \\
      &=& I.
    \end{eqnarray*}
  \end{proof}
\end{exercise}

%4.6
\begin{exercise}[Boch sphere interpretation of rotations]
  One reason why the $R_n(\theta)$ operators are reffered to as rotation operators in the following fact, which you are to prove. Suppose a single qubit has a state represented by the Bloch vector $\lambda$. Then the effect of the rotation $R_n(\theta)$ on the state is to rotate it by an angle $\theta$ about the $n$ axis of the Bloch sphere. This fact explains the rather mysterious looking factor of two in the definition of the rotation matrices.
  \begin{proof}
    We will begin with a deeper dive into the Pauli rotations. For example $R_z(\theta)$ can be written as
    \begin{eqnarray*}
      R_z(2\theta)&=&\cos(\theta)(\ket{0}\bra{0}+\ket{1}\bra{1})+i\sin(\theta)(\ket{0}\bra{0}-\ket{1}\bra{1}) \\
      &=&(\cos(\theta)+i\sin(\theta))\ket{0}\bra{0}+(\cos(\theta)-i\sin(\theta))\ket{1}\bra{1} \\
      &=&e^{i\theta}\ket{0}\bra{0}+e^{-i\theta}\ket{1}\bra{1}.
    \end{eqnarray*}
    That is, $R_z(\theta)$ has an eigenvalue of $e^{i\theta/2}$ for $\ket{0}$ and an eigenvalue of $e^{-i\theta/2}$ for $\ket{1}$. In terms of the Bloch sphere this is what the eigenvalues are for the antipodal points $(0,0,1)$ and $(0,0,-1)$ respectively -- the points about which $R_z(\theta)$ rotates qubits. A similar calculation shows that the same holds true for each of $R_x(\theta)$ and $R_y(\theta)$.

    Now we turn our attention to $n\cdot\sigma$. We already know that $n$ is the Bloch representation of the eigenvector of $R_n(\theta)$. [Do we really?] By a similar calculation above we can see that $R_n(\theta)$ is a rotation about the $n$-axis.
  \end{proof}
\end{exercise}

%4.7
\begin{exercise}
  Show that $XYX=-Y$ and use this to prove that $XR_y(\theta)X=R_y(-\theta)$.
  \begin{proof}
    A straight-forward calculation shows that $XYX=-Y$. We then have
    \begin{eqnarray*}
      XR_y(2\theta)X&=&
      \cos(\theta)XIX+i\sin(\theta)XYX \\
      &=&\cos(\theta)I-i\sin(\theta)Y \\
      &=&\cos(-\theta)I+i\sin(-\theta)Y.
    \end{eqnarray*}
    Therefore $XR_y(\theta)X=R_y(-\theta)$.
  \end{proof}
\end{exercise}

%4.8
\begin{exercise}
  An arbitrary single qubit unitary operator can be written in the form
  \[U=\exp(i\alpha)R_n(\theta)\]
  for some real number $\alpha$ and $\theta$ and a real three-dimensional unit vector $n$.
  \begin{enumerate}
    \item Prove this fact.
    \item Find values for $\alpha,\theta$, and $n$ giving the Hadamard gate $H$.
    \item Find values for $\alpha,\theta$, and $n$ giving the phase gate
    \[S=\begin{bmatrix}
      1&0\\
      0&i
    \end{bmatrix}.\]
  \end{enumerate}
  \begin{proof}[Proof of 1]
    Let $\ket{x}$ and $\ket{y}$ be orthonormal eigenvectors of $U$. Then we can write
    \[U=\lambda_x\ket{x}\bra{x}+\lambda_y\ket{y}\bra{y},\]
    where $|\lambda_y|=1$ and $|\lambda_x|=1$. That means we can write $\lambda_x=e^{i\phi_x}$ and $\lambda_y=e^{i\phi_y}$ for some $\phi_x,\phi_y\in\BR$. Let $n$ be the Bloch representation of $\ket{x}$. By lemma \ref{lem:bloch} we know that $\ket{y}=-n$. This means that $\ket{x}$ and $\ket{y}$ are also eigenvectors of $R_n(\theta)$ for any $\theta$. Let us write out and simplify $R_n(\theta)$.
    \begin{eqnarray*}
      R_n(\theta)&=&\cos\left(\frac{\theta}{2}\right)I+i\sin\left(\frac{\theta}{2}\right)(n_xX+n_yY+n_zZ) \\
      &=& \cos\left(\frac{\theta}{2}\right)(\ket{x}\bra{x}+\ket{y}\bra{y})+i\sin\left(\frac{\theta}{2}\right)(\ket{x}\bra{x}-\ket{y}\bra{y}) \\
      &=& \left[\cos\left(\frac{\theta}{2}\right)
      +i\sin\left(\frac{\theta}{2}\right)\right]\ket{x}\bra{x}+\left[\cos\left(\frac{\theta}{2}\right)
      -i\sin\left(\frac{\theta}{2}\right)\right]\ket{y}\bra{y} \\
      &=&e^{i\frac{\theta}{2}}\ket{x}\bra{x}+e^{-i\frac{\theta}{2}}\ket{y}\bra{y}.
    \end{eqnarray*}
    Now choosing $\theta/2=\phi_x-\phi_y$ and $\alpha=\phi_x+\phi_y$, we have
    \begin{eqnarray*}
      e^{i\alpha}R_n(\theta)
      &=&e^{i\alpha}(e^{i\frac{\theta}{2}}\ket{x}\bra{x}+e^{-i\frac{\theta}{2}}\ket{y}\bra{y}) \\
      &=&e^{i\alpha}(e^{i(\phi_x-\phi_y)}\ket{x}\bra{x}+e^{i(\phi_y-\phi_x)}\ket{y}\bra{y}) \\
      &=& e^{i(\alpha+\phi_x-\phi_y)}\ket{x}\bra{x}+e^{i(\alpha+\phi_y-\phi_x)}\ket{y}\bra{y} \\
      &=& e^{i\phi_x}\ket{x}\bra{x}+e^{i\phi_y}\ket{y}\bra{y} \\
      &=& U.
    \end{eqnarray*}
  \end{proof}
  \begin{proof}[Solution for 2]
    The eigenvector of the Hadamard gate is
  \end{proof}
\end{exercise}

\begin{theorem}[$Z-Y$ decomposition for a single qubit]
  \label{thm:decompxz}
  Suppose $U$ is a unitary operation on a single qubit. Then there exist real numbers $\alpha,\beta,\gamma,$ and $\delta$ such that
  \[
    U=e^{i\alpha}R_z(\beta)R_x(\gamma)R_z(\delta).
  \]
\end{theorem}

%4.9
\begin{exercise}
  Explain why any single qubit unitary operator may be written in the form
  \[
    U=\begin{bmatrix}
      e^{i(\alpha-\beta/2-\delta/2)}\cos\frac{\gamma}{2}&-e^{i(\alpha-\beta/2+\delta/2)}\sin\frac{\gamma}{2} \\
      e^{i(\alpha+\beta/2-\delta/2)}\sin\frac{\gamma}{2}&e^{i(\alpha+\beta/2+\delta/2)}\cos\frac{\gamma}{2}
    \end{bmatrix}.
  \]
\end{exercise}

%4.10
\begin{exercise}[$Z-Y$ decomposition for a single qubit]
  Give a decomposition analogous to Theorem \ref{thm:decompxz}.
\end{exercise}

%4.11
\begin{exercise}
  Suppose $m$ an $n$ are non-parallel real unit vectors in three dimensions. Use Theorem \ref{thm:decomp-xz} to show that an arbitrary single qubit unitary $U$ may be written
  \[
    U=e^{i\alpha}R_n(\beta)R_m(\gamma)R_n(\delta),
  \]
  for appropriate choices of $\alpha,\beta,\gamma$, and $\delta$.
\end{exercise}

\begin{corollary}
  \label{cor:axbxc}
  Suppose $U$ is a unitary gate on a single qubit. Then there exist unitary operators $A,B,C$ on a single qubit such that $ABC=I$ and $U=e^{i\alpha}AXBXC$, where $\alpha$ is some overall phase factor.
\end{corollary}

%4.12
\begin{exercise}
  Give $A,B,C$, and $\alpha$ for the Hadamard gate.
\end{exercise}

%4.13
\begin{exercise}[Circuit Identities]
  It is useful to be able to simplify circuits by inspection, using well-known identities. Prove the following identities:
  \[
    HXH = Z;\quad HYH=-Y;\quad HZH=X.
  \]
\end{exercise}

%4.14
\begin{exercise}
  Use the previous exercise to show that $HTH=R_x(\pi/4)$, up to a global phase.
\end{exercise}

%4.15
\begin{exercise}[Composition of single qubit operations]
  The Bloch representation gives a nice way to visualize the effect of composing two rotations.
  \begin{enumerate}
    \item Prove that if a rotation through an angle $\beta_1$ about the axis $n_1$ is followed by a rotation through an angle $\beta_2$ about the axis $n_2$, then the overall rotation is through an angle $\beta_{12}$ about the axis $n_{12}$, given by
    \begin{eqnarray*}
      c_{12}&=&c_1c_2-s_1s_2n_1\cdot n_2 \\
      s_{12}n_{12}&=&s_1c_2n_1+c_1s_2n_2-s_1s_2n_2\times n_1,
    \end{eqnarray*}
    where $c_i=\cos(\beta_i/2)$, $s_i=\sin(\beta_i/2)$, $c_{12}=\cos(\beta_{12}/2)$, and $s_{12}=\sin(\beta_{12}/2)$.
    \item Show that if $\beta_1=\beta_2$ and $n_1=z$ these equations simplify to
    \begin{eqnarray*}
      c_{12}&=&c^2-s^2z\cdot n_2 \\
      s_{12}n_{12}&=&sc(z+n_2)-s^2n^2\times z,
    \end{eqnarray*}
    where $c=c_1$ and $s=s_1$.
  \end{enumerate}
\end{exercise}

\setcounter{exercise}{24}
\begin{exercise}[Fredkin gate construction]
  Recall that the Fredkin (controlled-swap) gate performs the transform
  \[
    \begin{bmatrix}
      1&0&0&0&0&0&0&0 \\
      0&1&0&0&0&0&0&0 \\
      0&0&1&0&0&0&0&0 \\
      0&0&0&1&0&0&0&0 \\
      0&0&0&0&1&0&0&0 \\
      0&0&0&0&0&0&1&0 \\
      0&0&0&0&0&1&0&0 \\
      0&0&0&0&0&0&0&1 \\
    \end{bmatrix}.
  \]
  \begin{enumerate}
    \item Give a quantum circuit which uses three Toffoli gates to construct the Fredkin gate (Hint: think of the swap gate construction -- you can control each gate, one at a time).
    \item Show that the first and last Toffoli gates can be replaced by CNOT gates.
    \item Now replace the middle Toffoli gate with the circuit in Figure 4.8 to obtain a Fredkin gate construction using only six two-qubit gates.
    \item Can you come up with an even simpler construction, with only five two-qubit gates?
  \end{enumerate}
  \begin{proof}[Proof of 1]
    \[
          \begin{tikzpicture}[thick]
      % `operator' will only be used by Hadamard (H) gates here.
      % `phase' is used for controlled phase gates (dots).
      % `surround' is used for the background box.
      \tikzstyle{operator} = [draw,fill=white,minimum size=1.5em]
      \tikzstyle{phase} = [draw,fill,shape=circle,minimum size=5pt,inner sep=0pt]
      \tikzstyle{open} = [draw,shape=circle,minimum size=5pt,inner sep=0pt,text width=.5cm,text centered]
      \tikzstyle{surround} = [fill=blue!10,thick,draw=black,rounded corners=2mm]
      %
      \matrix[row sep=0.4cm, column sep=0.8cm] (circuit) {
      % First row.
      \coordinate (q1) {}; &
      \node[phase] (p11) {}; &
      \node[phase] (p12) {}; &
      \node[phase] (p13) {}; &
      \coordinate (end1); \\
      % Second row.
      \coordinate (q2) {}; &
      \node[phase] (p21) {}; &
      \node[open] (o22) {+}; &
      \node[phase] (p23) {}; &
      \coordinate (end2);\\
      % Third row.
      \coordinate (q3) {}; &
      \node[open] (o31) {+}; &
      \node[phase] (p32) {}; &
      \node[open] (o33) {$\oplus$}; &
      \coordinate (end3); \\
      };
      \begin{pgfonlayer}{background}
          % Draw lines.
          \draw[thick] (q1) -- (end1)  (q2) -- (end2) (q3) -- (end3) (p11) - (o31) (p12) -- (p32) (p13) -- (o33);
      \end{pgfonlayer}
      %
      \end{tikzpicture}
      \]
  \end{proof}
\end{exercise}
\end{document}
