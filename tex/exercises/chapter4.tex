\documentclass{article}
  \usepackage{amssymb}
  \usepackage{amsfonts}
  \usepackage{amsmath}
  \usepackage{amsthm}
  \usepackage[all]{xy}
  \usepackage{ifthen}
  \usepackage{makeidx}
  \usepackage{physics}
  \usepackage{exercise}
  \newcommand{\makeind}{\makeindex } \newcommand{\ind}[1]{ } \newcommand{\printind} {}
  %\usepackage{hyperref} \hypersetup{backref,colorlinks=true} \renewcommand{\ind}[1]{\index{#1}} \renewcommand{\makeind}{\makeindex} \renewcommand{\printind}{\printindex }
  \makeind
  \author{Korben Rusek}
  \title{Quantum Computing}
  \date{6-1-2018}
  \pagestyle{myheadings}
  \markright{Korben Rusek - Quantum Computing}
  \oddsidemargin 0.1in
  \evensidemargin 0.0in
  \textwidth 6.0in
  \begin{document}
  \maketitle
  \newcommand{\gindex}[2]{|#1\!:\!#2|}
  \newcommand{\lcm}{\textrm{lcm}}
  \newcommand{\irr}{\textrm{irr}}
  \newcommand{\sylp}{$Syl_{p}$}
  \newcommand{\phnt}[1]{$\phantom{1}^{#1}$}
  \newcommand{\gen}[1]{\langle#1\rangle}
  \newcommand{\BN}{\mathbb{N}}
  \newcommand{\BZ}{\mathbb{Z}}
  \newcommand{\BQ}{\mathbb{Q}}
  \newcommand{\BR}{\mathbb{R}}
  \newcommand{\BC}{\mathbb{C}}
  \newcommand{\BF}{\mathbb{F}}
  \newcommand{\CF}{\mathcal{F}}
  \newcommand{\CQ}{\mathcal{Q}}
  \newcommand{\fa}{\mathfrak{a}}
  \newcommand{\fb}{\mathfrak{b}}
  \newcommand{\fp}{\mathfrak{p}}
  \newcommand{\fq}{\mathfrak{q}}
  \newcommand{\fm}{\mathfrak{m}}
  \newcommand{\FN}{\mathfrak{N}}
  \newcommand{\FR}{\mathfrak{R}}
  \newcommand{\set}[1]{\{#1\}}
  \newcommand{\trv}{\set{1}}
  \newcommand{\Aut}{\mathrm{Aut}}
  \newcommand{\End}{\mathrm{End}}
  \newcommand{\Ker}{\mathrm{Ker}}
  \newcommand{\chr}{\mathrm{char}}

  \theoremstyle{definition}
  \newtheorem{theorem}{Theorem}[section]
  \newtheorem{definition}[theorem]{Definition}
  \newtheorem{lemma}[theorem]{Lemma}
  \newtheorem{exercise}{Exercise}[section]

  \setcounter{section}{4}
  \setcounter{subsection}{1}

\subsection{Single qubit operators}

\begin{exercise}
  In Exercise 2.11, you computed the eigenvectors of the Pauli matrices. Find the points on the Bloch sphere which correspond to the normalized eigenvectors of the different Pauli matrices.
  \begin{proof}

  \end{proof}
\end{exercise}

\begin{lemma}
  Let $\ket{x}$ and $\ket{y}$ be orthogonal then the Bloch representations are antipodal. \label{lem:bloch}
  \begin{proof}

  \end{proof}
\end{lemma}

\begin{exercise}
  Let $x$ be a real number and $A$ a matrix such that $A^2=I$. Show that
  \[\exp(Ax)=\cos(x)I+i\sin(x)A.\]
  User this result to verify Equations $(4.4)$ through $(4.6)$.
  \begin{proof}

  \end{proof}
\end{exercise}

\begin{exercise}
  Show that, up to a global phase, the $\pi/8$ gate satisfies $T=R_x(\pi/4)$.
  \begin{proof}

  \end{proof}
\end{exercise}

\begin{exercise}
  Express the Hadamard gate, $H$, as a product of $R_x$ and $R_z$ rotations and $e^{i\phi}$ for some $\phi$.
  \begin{proof}

  \end{proof}
\end{exercise}

If $n=(n_x, n_y, n_z)\in\BR^3$ is a real unit vector in three dimensions then we genearlize the previous definitions by defining a rotation by $\theta$ about the $n$ axis by the equation
\begin{equation}
  R_n(\theta) = \exp(i\theta n\cdot\sigma/2)
  =\cos\left(\frac{theta}{2}\right)I
  -i\sin\left(\frac{\theta}{2}\right)(n_xX+n_yY+n_zZ),\label{eq:rn}
\end{equation}
where $\sigma$ denotes the three component vector $(X,Y,Z)$ of Pauli matrices.

\begin{exercise}
  Prove that $(n\cdot\sigma)^2=I$, and use this to verify equation \ref{eq:rn}.
  \begin{proof}

  \end{proof}
\end{exercise}

\begin{exercise}[Boch sphere interpretation of rotations]
  One reason why the $R_n(\theta)$ operators are reffered to as rotation operators in the following fact, which you are to prove. Suppose a single qubit has a state represented by the Bloch vector $\lambda$. Then the effect of the rotation $R_n(\theta)$ on the state is to rotate it by an angle $\theta$ about the $n$ axis of the Bloch sphere. This fact explains the rather mysterious looking factor of two in the definition of the rotation matrices.
\end{exercise}

\begin{exercise}
  Show that $XYX=-Y$ and use this to prove that $XR_y(\theta)X=R_y(-\theta)$.
  \begin{proof}

  \end{proof}
\end{exercise}

%4.8
\begin{exercise}
  An arbitrary single qubit unitary operator can be written in the form
  \[U=\exp(i\alpha)R_n(\theta)\]
  for some real number $\alpha$ and $\theta$ and a real three-dimensional unit vector $n$.
  \begin{enumerate}
    \item Prove this fact.
    \item Find values for $\alpha,\theta$, and $n$ giving the Hadamard gate $H$.
    \item Find values for $\alpha,\theta$, and $n$ giving the phase gate
    \[S=\begin{bmatrix}
      1&0\\
      0&i
    \end{bmatrix}.\]
  \end{enumerate}
  \begin{proof}[Proof of 1]
    Let $\ket{x}$ and $\ket{y}$ be orthonormal eigenvectors of $U$. Then we can write
    \[U=\lambda_x\ket{x}\bra{x}+\lambda_y\ket{y}\bra{y},\]
    where $|\lambda_y|=1$ and $|\lambda_x|=1$. That means we can write $\lambda_x=e^{i\phi_x}$ and $\lambda_y=e^{i\phi_y}$ for some $\phi_x,\phi_y\in\BR$. Let $n$ be the Bloch representation of $\ket{x}$. By lemma \ref{lem:bloch} we know that $\ket{y}=-n$. This means that $\ket{x}$ and $\ket{y}$ are also eigenvectors of $R_n(\theta)$ for any $\theta$. Let us write out and simplify $R_n(\theta)$.
    \begin{eqnarray*}
      R_n(\theta)&=&\cos\left(\frac{\theta}{2}\right)I+i\sin\left(\frac{\theta}{2}\right)(n_xX+n_yY+n_zZ) \\
      &=& \cos\left(\frac{\theta}{2}\right)(\ket{x}\bra{x}+\ket{y}\bra{y})+i\sin\left(\frac{\theta}{2}\right)(\ket{x}\bra{x}-\ket{y}\bra{y}) \\
      &=& \left[\cos\left(\frac{\theta}{2}\right)
      +i\sin\left(\frac{\theta}{2}\right)\right]\ket{x}\bra{x}+\left[\cos\left(\frac{\theta}{2}\right)
      -i\sin\left(\frac{\theta}{2}\right)\right]\ket{y}\bra{y} \\
      &=&e^{i\frac{\theta}{2}}\ket{x}\bra{x}+e^{-i\frac{\theta}{2}}\ket{y}\bra{y}.
    \end{eqnarray*}
    Now choosing $\theta/2=\phi_x-\phi_y$ and $\alpha=\phi_x+\phi_y$, we have
    \begin{eqnarray*}
      e^{i\alpha}R_n(\theta)
      &=&e^{i\alpha}(e^{i\frac{\theta}{2}}\ket{x}\bra{x}+e^{-i\frac{\theta}{2}}\ket{y}\bra{y}) \\
      &=&e^{i\alpha}(e^{i(\phi_x-\phi_y)}\ket{x}\bra{x}+e^{i(\phi_y-\phi_x)}\ket{y}\bra{y}) \\
      &=& e^{i(\alpha+\phi_x-\phi_y)}\ket{x}\bra{x}+e^{i(\alpha+\phi_y-\phi_x)}\ket{y}\bra{y} \\
      &=& e^{i\phi_x}\ket{x}\bra{x}+e^{i\phi_y}\ket{y}\bra{y} \\
      &=& U.
    \end{eqnarray*}
  \end{proof}
\end{exercise}

\begin{theorem}[$Z-Y$ decomposition for a single qubit]

\end{theorem}

\end{document}
