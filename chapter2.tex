\documentclass{article}
\usepackage{amssymb}
\usepackage{amsfonts}
\usepackage{amsmath}
\usepackage{amsthm}
\usepackage[all]{xy}
\usepackage{ifthen}
\usepackage{makeidx}
\usepackage{physics}
\usepackage{exercise}
\newcommand{\makeind}{\makeindex } \newcommand{\ind}[1]{ } \newcommand{\printind} {}
%\usepackage{hyperref} \hypersetup{backref,colorlinks=true} \renewcommand{\ind}[1]{\index{#1}} \renewcommand{\makeind}{\makeindex} \renewcommand{\printind}{\printindex }
\makeind
\author{Korben Rusek}
\title{Quantum Computing}
\date{9-5-2007}
\pagestyle{myheadings}
\markright{Korben Rusek - Quantum Computing}
\oddsidemargin 0.1in
\evensidemargin 0.0in
\textwidth 6.0in
\begin{document}
\maketitle
\newcommand{\gindex}[2]{|#1\!:\!#2|}
\newcommand{\lcm}{\textrm{lcm}}
\newcommand{\irr}{\textrm{irr}}
\newcommand{\sylp}{$Syl_{p}$}
\newcommand{\phnt}[1]{$\phantom{1}^{#1}$}
\newcommand{\gen}[1]{\langle#1\rangle}
\newcommand{\BN}{\mathbb{N}}
\newcommand{\BZ}{\mathbb{Z}}
\newcommand{\BQ}{\mathbb{Q}}
\newcommand{\BR}{\mathbb{R}}
\newcommand{\BC}{\mathbb{C}}
\newcommand{\BF}{\mathbb{F}}
\newcommand{\CF}{\mathcal{F}}
\newcommand{\CQ}{\mathcal{Q}}
\newcommand{\fa}{\mathfrak{a}}
\newcommand{\fb}{\mathfrak{b}}
\newcommand{\fp}{\mathfrak{p}}
\newcommand{\fq}{\mathfrak{q}}
\newcommand{\fm}{\mathfrak{m}}
\newcommand{\FN}{\mathfrak{N}}
\newcommand{\FR}{\mathfrak{R}}
\newcommand{\set}[1]{\{#1\}}
\newcommand{\trv}{\set{1}}
\newcommand{\Aut}{\mathrm{Aut}}
\newcommand{\End}{\mathrm{End}}
\newcommand{\Ker}{\mathrm{Ker}}
\newcommand{\chr}{\mathrm{char}}

\theoremstyle{definition}
\newtheorem{theorem}{Theorem}[section]
\newtheorem{definition}[theorem]{Definition}
\newtheorem{lemma}[theorem]{Lemma}
\newtheorem{exercise}{Exercise}[section]

\setcounter{section}{1}
\section{Linear Algebra}
\begin{lemma}
  Let $A$ be a non-singular linear operator.
  If all the eigenvalues of $A$ are $\pm 1$ then $A^2=I$
  \begin{proof}
    Since $A$ is non-singular and normal with eigenvalues $\pm1$ then we
    can write $A=\sum\lambda_i\ket{i}\bra{i}$ with $\ket{i}$ spanning the
    vector space. Then we
    have
    \begin{align*}
      A^2&=\left(\sum_i\lambda_i\ket{i}\bra{i}\right)\left(\sum_j\lambda_j\ket{j}\bra{j}\right) \\
      &=\sum_i\sum_j\lambda_i\ket{i}\bra{i}\lambda_j\ket{j}\bra{j} \\
      &=\sum_i\sum_j\lambda_i\lambda_j\ket{i}\bra{i}\ket{j}\bra{j} \\
      &=\sum_i\sum_j\lambda_i\lambda_j\ket{i}\delta_{i,j}\bra{j} \\
      &=\sum_i\sum_j\lambda_i\lambda_j\delta_{i,j}\ket{i}\bra{j} \\
      &=\sum_i\lambda_i^2\ket{i}\bra{i} \\
      &=\sum_i\ket{i}\bra{i} \\
      &=I
    \end{align*}
  \end{proof}
\end{lemma}

\begin{lemma}
  Let $A=\sum_x\ket{x}\bra{x}-\sum_y\ket{y}\bra{y}$, where $\{\ket{x},\ket{y}\}_{x,y}$ form an orthonormal basis.
  Then
  \[
    f(\theta A) = \frac{f(\theta)+f(-\theta)}{2}I + \frac{f(\theta)-f(-\theta)}{2}A.
  \]
  \begin{proof}
    We can write $A=\sum_x\ket{x}\bra{x}-\sum_y\ket{y}\bra{y}$.
    Let $X=\sum_x \ket{x}\bra{x}$ and
    $Y=\sum_y \ket{y}\bra{y}$. That means that
    $I=X+Y$ and $A=X-Y$. Then we have
    \begin{eqnarray*}
      f(\theta)A&=&\sum_xf(\theta)\ket{x}\bra{x}+\sum_yf(-\theta)\ket{y}\bra{y} \\
      &=&f(\theta)\sum_x\ket{x}\bra{x}+f(-\theta)\sum_y\ket{y}\bra{y} \\
      &=&f(\theta)X+f(-\theta)Y \\
      &=&\frac{f(\theta)}{2}X+\frac{f(\theta)}{2}Y+\frac{f(\theta)}{2}X-\frac{f(\theta)}{2}Y+\frac{f(-\theta)}{2}Y+\frac{f(-\theta)}{2}Y+\frac{f(-\theta)}{2}X-\frac{f(-\theta)}{2}X \\
      &=&\frac{f(\theta)}{2}(X+Y)+\frac{f(-\theta)}{2}(X+Y)+\frac{f(\theta)}{2}(X-Y)-\frac{f(-\theta)}{2}(X-Y) \\
      &=&\frac{f(\theta)+f(-\theta)}{2}I+\frac{f(\theta)-f(-\theta)}{2}A
    \end{eqnarray*}
  \end{proof}
\end{lemma}
\setcounter{exercise}{43}
\begin{exercise}
  Suppose that $A$ is invertible and that $\{A,B\}=[A,B]=0$. Show that $B$ is 0.
  \begin{proof}
    $[A,B]=0$ tells us that $AB=BA$ and $\{A,B\}=0$ tells us that $AB=-BA$. Therefore we
    know that $BA=-BA$. Now multiplying on the right side by $A^{-1}$, we get $B=-B$. Thus
    $-2B=0$ which implies that $B=0$.
  \end{proof}
\end{exercise}

\begin{exercise}
  Show that $[A,B]^\dagger=[B^\dagger,A^\dagger]$.
  \begin{proof}
    \begin{eqnarray*}
      [A,B]^\dagger&=&\left(AB-BA\right)^\dagger \\
      &=&B^\dagger A^\dagger - A^\dagger B^\dagger \\
      &=&[B^\dagger,A^\dagger ].
    \end{eqnarray*}
  \end{proof}
\end{exercise}

\begin{exercise}
  Show that $[A,B]=-[B,A]$.
  \begin{proof}
    \begin{eqnarray*}
      [A,B]&=&AB-BA \\
      &=&- (BA-AB) \\
      &=&-[B,A]
    \end{eqnarray*}
  \end{proof}
\end{exercise}

\begin{exercise}
  Suppose that $A$ and $B$ are Hermitian. Show that $i[A,B]$ is Hermitian.
  \begin{proof}
    Suppose that $A=A^\dagger$ and $B=B^\dagger$. Then we have
    \begin{eqnarray*}
      \left(i[A,B]\right)^\dagger&=&-i[B^\dagger,A^\dagger] \\
      &=&-i[B,A]=i[A,B].
    \end{eqnarray*}
    Therefore $i[A,B]$ is Hermitian.
  \end{proof}
\end{exercise}

\begin{lemma}
  Let $A$ be a diagonalizable matrix. Write $A=\sum_i\lambda_i\ket{i}$. Then $\sqrt{A^\dagger A}=\sum_i|\lambda_i|\ket{i}\bra{i}$.
  \begin{proof}
    The proof is pretty straight forward. We start with
    \begin{eqnarray*}
      A^\dagger A &=& \left(\sum_i \lambda_i^*\ket{i}\bra{i}\right)\left(\sum_i\lambda_i\ket{i}\bra{i}\right) \\
      &=&\sum_i|\lambda_i|^2\ket{i}\bra{i}.
    \end{eqnarray*}
    Therefore $\sqrt{A^\dagger A}=\sum_i|\lambda_i|\ket{i}\bra{i}$.
  \end{proof}
\end{lemma}

\begin{exercise}
What is the polar decomposition of a positive matrix, $P$? Of a unitary matrix, $U$? or a Hermitian matrix, $H$?
\begin{proof}[Positive matrix]
  By the above lemma, for a positive matrix, $P$, $\sqrt{P^\dagger P}=P$. Therefore we have $IP$ or $PI$ as the polar decompositions.
\end{proof}
\begin{proof}[Unitary matrix]
  Suppose $V$ is unitary. By the above lemma $\sqrt{V^\dagger V}=I$. Therefore $V=VI=IV$ is the polar decomposition.
\end{proof}
\begin{proof}[Hermitian matrix]
  Suppose that $H$ is Hermitian. Write $H=\sum_i\lambda_i\ket{i}\bra{i}$. Then $J=\sum_i|\lambda_i|\ket{i}\bra{i}$. Let $a_i$ be defined as $\lambda_i/|\lambda_i|$ when $\lambda_i\ne0$ and $1$ when $\lambda_i=0$. Then $U=\sum_iv_i\ket{i}\bra{i}$.
\end{proof}
\end{exercise}

\begin{exercise}
  Express the polar decomposition of a normal matrix in the outer product representation.
  \begin{proof}
    The solution is similar to what we see in the Hermitian version of the above exercise. $J$ has eigenvalues that are the absolute value of the eigenvalues of $H$. $U$ would have eigenvalues that are the unit vectors of the eigenvalues of $H$ (or 1 when eigenvalues are 0).
  \end{proof}
\end{exercise}

\begin{exercise}
  Find the left and right polar decompositions of the matrix
  \[
  P=\begin{bmatrix}
    1&0 \\
    1&1
  \end{bmatrix}.
  \]
  \begin{proof}
    It is easy to see that the eigenvalues of $P$ is $1$ repeated. Therefore $P$ is positive. Therefore $U=I$ and $J=K=P$.
  \end{proof}
\end{exercise}

\begin{definition}
  We define $H$ to be the Hadamard matrix.
  \[
    H=\frac{1}{\sqrt2}\begin{bmatrix}
    1&1 \\
    1&-1
  \end{bmatrix}.
    \]
\end{definition}

\begin{exercise}
  Verify that th Hadamard gate $H$ is unitary.
  \begin{proof}
    \begin{eqnarray*}
      H^2&=&\frac{1}{2}\begin{bmatrix}
        2&0 \\
        0&2
      \end{bmatrix} = I.
    \end{eqnarray*}
    It is clear that $H^\dagger = H$. Thus $H\dagger H=I$ and $H$ is unitary.
  \end{proof}
\end{exercise}

\begin{exercise}
  Verify that $H^2=I$.
  \begin{proof}
    This was shown in the previous exercise.
  \end{proof}
\end{exercise}

\begin{exercise}
  What are the eigenvalues and eigenvectors of $H$?
  \begin{proof}
    To find the eigenvalues of $H$ we solve the  polynomial equation
    \begin{eqnarray*}
      0&=&(1-\lambda)(-1-\lambda)-1 \\
      &=&(\lambda-1)(\lambda + 1)-1 \\
      &=&\lambda^2-2.
    \end{eqnarray*}
    Thus our eigenvalues are $\lambda=\pm\sqrt2$. To find the eigenvectors we solve the system:
    \begin{eqnarray*}
      x+y&=&\frac{\lambda}{\sqrt2} x \\
      x-y&=&\frac{\lambda}{\sqrt2} y
    \end{eqnarray*}
    Since $(0,1)$ is clearly not an eigenvector we can assume that $x=1$. For $\lambda=1$ we have
    $y=\frac{1-\sqrt2}{\sqrt2}$. For $\lambda=-1$ we get $y=\frac{-1+\sqrt2}{\sqrt2}$. Therefore we have the eigenvectors $(\sqrt2,1-\sqrt2)$ and $(\sqrt2,\sqrt2-1)$ with eigenvalues $1$ and $-1$ respectively.
  \end{proof}
\end{exercise}

\begin{exercise}
  Suppose $A$ and $B$ are commuting Hermitian operators. Prove that $exp(A)exp(B)=exp(A+B)$.
  \begin{proof}
    We can write $A=\sum_ia_i\ket{i}\bra{i}$ and $B=\sum_ib_i\ket{i}\bra{i}$ for some orthonormal basis $\ket{i}$. Then we have
    \[exp(A)=\sum_ie^{a_i}ket{i}\bra{i},\]
    \[exp(B)=\sum_ie^{b_i}ket{i}\bra{i},\]
    and so
    \begin{eqnarray*}
      exp(A)exp(B)&=&\sum_ie^{a_i}e^{b_i}ket{i}\bra{i} \\
       &=&\sum_ie^{a_i+b_i}ket{i}\bra{i} \\
       &=&exp(A+B).
    \end{eqnarray*}
  \end{proof}
\end{exercise}

\begin{definition}
  \[
    U(t_1,t_2)=exp\left[\frac{-iH(t_2-t_1)}{\hbar}\right].
    \]
\end{definition}

\begin{exercise}
  Prove that $U(t_1,t_2)$ is unitary.
  \begin{proof}
    This is simple as $exp(x)$ is non-zero, so $U$ is unitary.
  \end{proof}
\end{exercise}

\begin{exercise}
  Use spectral decomposition to show that $K=-i\log(U)$ is Hermitian for any unitary $U$, and thus $U=\exp(iK)$ for some Hermitian $K$.
  \begin{proof}
    Let $U$ be unitary. Then we can write $U=\sum\alpha_\phi\ket{\phi}\bra{\phi}$ for some orthonormal basis $\bra{\phi}$ and $\alpha_\phi\ne0$. That means that $K=\sum-i\log(\alpha_\phi)\ket{\phi}\bra{\phi}$. Since $U$ is unitary then $\alpha_\phi=e^{i\theta_\phi}$ for some real $\theta_\phi$. Therefore we can simplify $K$ to
    \[
      K=\sum -i(i\theta_phi)\ket{\phi}\bra{\phi}=\sum\theta_\phi\ket{\phi}\bra{\phi}.
    \]
    Now sime $\theta_\phi$ is real then $K$ is Hermitian. Therefore any unitary operator is $\exp(iK)$ for some Hermitian $K$.
  \end{proof}
\end{exercise}

\setcounter{exercise}{57}

\begin{exercise}
  Suppose we prepare a quantum system in an eigenstate $\ket\psi$ of some observable $M$, with corresponding eigenvalue $m$. What is the average observed value of $M$, and the standard deviation?
  \begin{proof}
    The average observed value of $M$ is given by $E(M)=\bra{\psi}M\ket\psi$. Then we have
    \begin{eqnarray*}
      E(M)&=&\bra\psi M\ket\psi \\
      &=&m\bra\psi\ket\psi \\
      &=&m.
    \end{eqnarray*}
    The standard deviation squared is given by
    \[[\Delta(M)]^2=\langle M^2\rangle-\langle M\langle^2.\]
    We already saw that $\langle M\rangle=m$. And so we have
    \begin{eqnarray*}
      \langle M^2\rangle &=& \bra{\psi}{M^2}\ket{\psi} \\
      &=& \bra\psi mM\ket\psi \\
      &=& m^2\braket\psi \\
      &=& m^2.
    \end{eqnarray*}
    Therefore the standard deviation is $0$.
  \end{proof}
\end{exercise}

\begin{exercise}
  Suppose we have a qubit in the state $\ket0$, and we measure the observable $X$. What is the average value of $X$? What is the standard deviation of $X$?
  \begin{proof}
    The average value is given by $\bra{0}X\ket{0}=\bra0\ket{1}=0$.

    The standard deviation squared is
    \[
      \bra{0}X^2\ket{0}=\bra{0}\ket{0}=1.
    \]
  \end{proof}
\end{exercise}

\begin{exercise}
  Show that $v\cdot\sigma$ has eigenvalues $\pm1$ and that the projectors onto the correspnding eigenspaces are given by $P_\pm=(I\pm v\cdot\sigma)/2$.
  \begin{proof}
    We have already seen that $v\cdot\sigma$ has eigenvalues $\pm1$. It is easy to verify that the eigenvectors of $v\cdot\sigma$ are
    \[e_\pm=\begin{bmatrix}
      1\pm c \\
      a\pm bi
    \end{bmatrix}.\]
    Furthermore $P\pm e_\pm = e_\pm.$ Therefore $P_\pm$ are the projectors.
  \end{proof}
\end{exercise}

\begin{exercise}
  Calculate the probability of obtaining $+1$ for a measurement of $v\cdot\sigma$, given that the state prior to measurement is $\ket0$. What is the state of the system after the measurement?
  \begin{proof}
    The value $p(+1)$ is given by $\bra0P_+\ket0$. This gives
  $\frac{1}{2}\bra0\ket{\begin{bmatrix}
    1+c \\
    a+bi
  \end{bmatrix}}=\frac{1+c}{2}.$

  After measurement the state of the system is \[\frac{P_+\ket0}{\sqrt{p(+1)}}=\begin{bmatrix}
    1+c \\
    a+bi
  \end{bmatrix}\sqrt\frac{2}{1+c}\]
  \end{proof}
\end{exercise}

\end{document}
